% --------------------------------------------------------------
% This is all preamble stuff that you don't have to worry about.
% Head down to where it says "Start here"
% --------------------------------------------------------------
 
\documentclass[12 pt]{article}
\usepackage{amsmath, amssymb, amsthm}

\theoremstyle{definition}
% May want theorems numbered by chapter
\newtheorem*{thm}{Theorem}
\newtheorem*{prob}{Problem}
\newtheorem*{ex}{Exercise}

\newcommand{\R}{\mathbb{R}}
\newcommand{\N}{\mathbb{N}}
\newcommand{\Z}{\mathbb{Z}}

\newcommand\m[1]{\begin{pmatrix}#1\end{pmatrix}}
\newcommand\x[2]{\noindent\textbf{Graded on {#1} - $\times$ - resubmit by {#2}}\\}
\newcommand\ch[1]{\noindent\textbf{Graded on {#1} - $\checkmark$}\\}
\newcommand\rev[1]{\noindent\textbf{Last revised on {#1}}\\}
 
\begin{document}
 
% --------------------------------------------------------------
%                         Start here
% --------------------------------------------------------------
 
\title{Weekly Homework X}%replace X with the appropriate number
\author{Jeff Ford\\ %replace with your name
Linear Algebra\\
Collaborators: } %if necessary, include the names of everyone you worked with on this.
 
\maketitle
 
\begin{prob}\\ 
Delete this text and write theorem statement here.
\end{prob}
\rev{2/19/18} %Add another of these whenever you submit a proof or it's revisisons. Do not delete this or any of the grades that show up below it.

If you just type stuff, it shows up as you'd expect. If you need special characters, underlining, overlining, italics, bold, Greek letters, of anything else, the commands are all in the Reference Sheet in Moodle.

If you want some math in the middle of a line, just put it in like this $x^2$ or $x_2$. Notice though, that if you do $x_21$ it doesn't look right. Instead, try $x_{21}$.

Suppose you need lots of equations lined up over several lines. Try this:

\begin{align*}
f(x) &= x+y\\
&= x-y\\
&=0
\end{align*}

If you want to use the proof environment, it's here. It ends with a nice box to show that the proof is over.

\begin{proof}
Look, here's a matrix.
$$A = \m{1&0&0\\0&1&0\\0&0&1}$$

There's a vector too!

$$\vec{v} = \m{1\\2\\3}$$
\end{proof}

% --------------------------------------------------------------
%     You don't have to mess with anything below this line.
% --------------------------------------------------------------
 
\end{document}
