\documentclass[12 pt]{article}
\usepackage{amsmath, amssymb, amsthm}
\usepackage{hyperref}
\usepackage{comment}
\usepackage{graphicx}

\theoremstyle{definition}
\newtheorem*{thm}{Theorem}
\newtheorem*{prob}{Problem}
\newtheorem*{exercise}{Exercise}
\newtheorem*{lemma}{Lemma}

\newcommand{\C}{\mathbb{C}}
\newcommand{\R}{\mathbb{R}}
\newcommand{\N}{\mathbb{N}}
\newcommand{\Z}{\mathbb{Z}}

\newcommand\m[1]{\begin{pmatrix}#1\end{pmatrix}}
\newcommand\x[2]{\noindent\textbf{Graded on {#1} - $\times$ - resubmit by {#2}}\\}
\newcommand\ch[1]{\noindent\textbf{Graded on {#1} - $\checkmark$}\\}
\newcommand\rev[1]{\noindent\textbf{Last revised on {#1}}\\}
 
\begin{document}
 
% --------------------------------------------------------------
%                         Start here
% --------------------------------------------------------------
 
\title{Problem number X}%replace X with the appropriate number
\author{Jeff Ford\\ %replace with your name
Your Course\\ %replace with the name of the course
Collaborators: } %if necessary, include the names of everyone you worked with on this.
 
\maketitle
\rev{2/19/20} %This tells me the last time you editted the document. Add another of these whenever you submit a proof or it's revisions. Do not delete this or any of the grades that show up below it. We keep track of how many submissions you need and how quickly I expect revisions.

\begin{thm} 
Delete this text and write the statement you are proving here. You can duplicate this theorem environment if you need to prove multiple theorems in a single problem. 
\end{thm}

\begin{lemma}
If you have a smaller result that needs to be proven before your major result, that is called a Lemma. You can use this environment to set up a lemma. Delete this environment if your proof doesn't need a lemma.
\end{lemma}



\begin{proof}
Begin your proof here. You'll want to separate proof environment for each part of your writeup. So if you are using multiple theorems, or a lemma, you'll have a \begin{verbatim}\begin{proof}\end{verbatim}and \begin{verbatim}\end{proof}\end{verbatim} after each one, with your writing in the middle.


If you just type stuff, it shows up as you'd expect. If you need special characters, underlining, overlining, italics, bold, Greek letters, or anything else, the commands are all  \url{https://oeis.org/wiki/List_of_LaTeX_mathematical_symbols}.\\

More so than memorizing commands, the most important thing you can do is \textbf{PROOFREAD} your document. Before you ever submit a document, hit Recomplile, and read the PDF that is produced. Look for mistakes.\\

If you want some math in the middle of a line, just put it in like this $x^2$ or $x_2$. Notice though, that if you do $x_21$ it doesn't look right. Instead, try $x_{21}$. The same thing applies to $x^21$ versus $x^{21}$.\\

Sometimes we want the math to be bigger, so we use $$x^2$$ to center it and make it larger. Especially useful for things like $\sum_{i=1}^n i$ versus $$\sum_{i=1}^n i.$$

When we put a blank line between two paragraphs, we get an automatic indent, but we don't get a blank line.

If we want some small stuff on a line, and no indent on the next line, just end a line with \\
then the next line is just below. Or you can put two backslashes $(\backslash\backslash)$ \\

at the end of a line, then a blank line, and then you get a new line, and a blank line between.

Maybe you want to make a list? Looks at the difference between 
$$1,2,...,n,... \mbox{ and } 1,2,\ldots,n,\ldots$$
Using the right command makes it easier to read.

If you want a bulleted list,
\begin{itemize}
\item Item 1
\item Item 2
\end{itemize}
the itemize environment will do the trick. You can use enumerate
\begin{enumerate}
\item
\item
\end{enumerate}
if you want it pre-numbered.

We can write functions in the usual way, $f(x) = 2x +1$, but we sometimes just need to state the domain and range, using something like $$f: \R \rightarrow \R.$$

There's one example an arrow above, but you might need $\leftarrow$, $\Rightarrow$, or $\iff$. These, and many others, are available on the reference page.

Suppose you need lots of equations lined up over several lines. Try this:

\begin{align*}
f(x) &= x+y\\
&= x-y\\
&=0
\end{align*}

If you need larger systems of equations, you can use dots like this example.

\begin{align*}
    a_{11}x_1 + a_{12}x_2 + \cdots + a_{1n}x_n &= c_1\\
    a_{21}x_1 + a_{22}x_2 + \cdots + a_{2n}x_n &= c_2\\
    &\vdots\\
    a_{m1}x_1 + a_{m2}x_2 + \cdots + a_{mn}x_n &= c_m\\
\end{align*}


Look, here's a matrix. I added a particular command for this to save you some time.
$$A = \m{1&0&0\\0&1&0\\0&0&1}$$

You might want dots in the middle of an arbitrary matrix too
$$\m{a_{11}&a_{12}&\cdots& a_{1n}\\
a_{12}&a_{22}&\cdots a_{2n}\\
\vdots&\vdots&\ddots&\vdots\\
a_{n1}&a_{n2}&\cdots& a_{nn}}$$

Maybe you're doing linear algebra, and you want to distinguish scalars from vectors. Using the $\vec{v}$ notation can make it clear you are working with a vector.

The other thing is that you'll often want to talk about sets. The curly bracket is already a reserved symbol, but we can fix that by putting a backslash in front of it. We can also use the mbox command to put regular text inside of math mode.
$$\R \{x:x \mbox{ is a real number}\}$$
You might need to make bigger notation around things. Compare these cases.
$$(\frac{n}{k})\mbox{ or }\left(\frac{n}{k}\right)$$
$$\{\frac{n}{k}\}\mbox{ or }\left\{\frac{n}{k}\right\}$$

Maybe you don't want the fraction bar in there. 
$${n \choose k}$$

Finally, I put in some notation for commonly labelled sets, to save you time.
$$\N \subseteq \Z \subseteq \R \subseteq \C$$

\end{proof}

% --------------------------------------------------------------
%     You don't have to mess with anything below this line.
% --------------------------------------------------------------
 
\end{document}
